\documentclass[12pt]{article}
\usepackage{times}
\usepackage[english]{babel}
\usepackage[utf8x]{inputenc}
\usepackage[colorinlistoftodos]{todonotes}
\usepackage[margin=1in]{geometry}
\usepackage{graphicx}
\usepackage{epstopdf}
\usepackage{cite}
\usepackage{listings}
\usepackage{dtklogos}
\usepackage{wrapfig}
\usepackage{subfigure}
\usepackage{amsmath}
\usepackage{amsthm}
\usepackage{amssymb}
\usepackage{amscd}
\usepackage{caption}
\usepackage{etoolbox}
\usepackage{fancyhdr}
\usepackage{stackengine}
\usepackage[export]{adjustbox}
\patchcmd{\thebibliography}{\section*{\refname}}{}{}{}
\usepackage[document]{ragged2e}    %This causes text to left align
\usepackage[colorlinks=true, linkcolor=black,citecolor=black,urlcolor=blue]{hyperref}
\bibliographystyle{IEEEtran}
\extrafloats{100}
\DeclareGraphicsRule{.tif}{png}{.png}{`convert #1 `dirname #1`/`basename #1 .tif`.png}

\title{MCHE 357: Lab 4}

\begin{document}
\lefthyphenmin3
\righthyphenmin4
% \pretolerance=2000
% \tolerance=500 
% \emergencystretch=10pt
%\raggedright     %Stops LaTeX from automatically hyphenating the right margin to fit better
%Combine this with \usepackage[document]{ragged2e} to get a text align left similar to natural MS Word


%-------------------------------------------------------------
%Start of Paper
%-------------------------------------------------------------

%%%%%%%%%%%%%%%%%%%%%%%%%%%%%%%%%%%%%%%%%%%%%%%%%%%%%%
%%%%%%%%%%%%%%%%%%%%%%% TITLE PAGE %%%%%%%%%%%%%%%%%%%%%%%%
%%%%%%%%%%%%%%%%%%%%%%%%%%%%%%%%%%%%%%%%%%%%%%%%%%%%%%

\begin{titlepage}

\newcommand{\HRule}{\rule{\linewidth}{0.5mm}} % Defines a new command for the horizontal lines, change thickness here

\center % Center everything on the page
 
%----------------------------------------------------------------------------------------
%	Heading Section
%----------------------------------------------------------------------------------------

\textsc{\LARGE University of Louisiana at Lafayette}\\[1.5cm] % Name of your university/college
\textsc{\Large Measurements and Instrumentation}\\[0.5cm] % Major heading such as course name
\textsc{\large MCHE 357}\\[0.5cm] % Minor heading such as course title

%----------------------------------------------------------------------------------------
%	Title Section
%----------------------------------------------------------------------------------------

\HRule \\[0.4cm]
{ \huge \bfseries Lab 4}\\[0.4cm] % Title of your document
\HRule \\[1.5cm]
 
%----------------------------------------------------------------------------------------
%	Author Section
%----------------------------------------------------------------------------------------

\begin{minipage}{0.4\textwidth}
\begin{flushleft} \large
\emph{Author:}\\
\textsc{Matthew J. Begneaud} \\% Your name
\end{flushleft}
\end{minipage}
~
\begin{minipage}{0.4\textwidth}
\begin{flushright} \large
\emph{Professor:} \\
\textsc{Dr. Mostafa A. Elsayed} % Supervisor's Name
\end{flushright}
\end{minipage}\\[1.5cm]

% If you don't want a supervisor, uncomment the two lines below and remove the section above
%\Large \emph{Author:}\\
%John \textsc{Smith}\\[3cm] % Your name

%----------------------------------------------------------------------------------------
%	Date Section
%----------------------------------------------------------------------------------------

{\textsc{\large \today}}\\[0.5cm] % Date, change the \today to a set date if you want to be precise


%----------------------------------------------------------------------------------------
%	Group Section
%----------------------------------------------------------------------------------------
% No group for this lab
%\textsc{\large Group:}\\[0.1cm]
%\textsc{Ronald Kisor}\\
%\textsc{Chandler Lagarde}\\
%\textsc{Somto Umeokafor}
%\\[0.5cm]

%----------------------------------------------------------------------------------------
%	Logo Section
%----------------------------------------------------------------------------------------

\includegraphics[width=5in]{UL_logo.jpg}\\[1cm] % Include a department/university logo - this will require the graphicx package
 
%----------------------------------------------------------------------------------------

\vfill % Fill the rest of the page with whitespace

\end{titlepage}

%%%%%%%%%%%%%%%%%%%%%%%%%%%%%%%%%%%%%%%%%%%%%%%%%%%%%%
%%%%%%%%%%%%%%%%%%%%%%% TABLE OF CONTENTS %%%%%%%%%%%%%%%%%%%
%%%%%%%%%%%%%%%%%%%%%%%%%%%%%%%%%%%%%%%%%%%%%%%%%%%%%%

\tableofcontents

\listoffigures

\bigskip


\section*{\fontsize{12}{12}\selectfont \large List of Symbols}
\addcontentsline{toc}{section}{List of Symbols} % Add for each section
$G$ = Gain\\
$f$ = Input Frequency (Hertz)\\
$f_{c}$ = Cutoff Frequency (Hertz)\\
$n$ = Butterworth Filter Order\\



\newpage

%%%%%%%%%%%%%%%%%%%%%%%%%%%%%%%%%%%%%%%%%%%%%%%%%%%%%%
%%%%%%%%%%%%%%%%%%%%%%% REPORT %%%%%%%%%%%%%%%%%%%%%%%%%%
%%%%%%%%%%%%%%%%%%%%%%%%%%%%%%%%%%%%%%%%%%%%%%%%%%%%%%


\section*{\fontsize{12}{12}\selectfont \large Introduction}
\addcontentsline{toc}{section}{Introduction} % Add for each section
This lab consisted of learning the  basics of Native Instrument's Lab View software for future labs. Two configurations were set up within the software. One was set up to measure pressure and temperature while converting the temperature from Celsius to Fahrenheit. The other was used to add two harmonic signals and use Fast-Fourier-Transform (FFT) analysis and Butterworth filters.


\section*{\fontsize{12}{12}\selectfont \large Theory}
\addcontentsline{toc}{section}{Theory} % Add for each section
Lab View is used to simulate real control systems, and consists of many optional devices and operations that can be used in the simulations. In control systems, typically data is read in from sensors and is then sent through different devices and controllers to give a desired output or feedback. Some example of operations is summing two or more inputs, converting data units, performing FFT analysis, passing signals through filters, and plotting data back to the user if information about a signal is required.
\bigskip

FFT is a computational method which performs a Fourier Transform by using computational tricks to speed up the calculation rather than try to actually solve differential equations. The FFT technique is implemented in order to measure the natural frequency of a system. This is done typically by returning a set of frequencies along with magnitudes that correspond to these frequencies. The frequencies with the highest magnitudes are considered the "dominant" frequencies of a system, and indicate which frequencies the system naturally vibrates with.
\bigskip

A common type of filter used in control systems is a Butterworth Filter, the equation of which is shown in Equation 1. This type of filter is implemented via an inverting amplifier. The purpose of this filter is to create a flat frequency response in the passband. These filters are typically used to flatten a signal which is irregular, which is useful in systems because data collected from sensors tends to be irregular and rough. The order $n$ of a Butterworth Filter dictates how flat the frequency response of the signal is.
\bigskip

\begin{equation}
G = \frac{1}{\sqrt{1 + (\frac{f}{f_{c}})^{2n}}}
\end{equation}
\bigskip


\section*{\fontsize{12}{12}\selectfont \large Procedure \& Analysis}
\addcontentsline{toc}{section}{Procedure \& Analysis} % Add for each section
In the first configuration, a random number generator was used to feed a number to a pressure gauge and a thermometer (in Celsius). The temperature was then passed through a formula node to change units to Fahrenheit. The output was then measured on a chart, and the simulation was run continuously by utilizing a while loop with a boolean switch for the conditional argument. The system diagram is shown in Figure 1 and the outputs and gauges can be seen in Figure 2.
\bigskip

\newpage

\begin{figure}[h!] %  figure placement: here, top, bottom, or page
   \centering
   \includegraphics[width=\linewidth]{lab_4a.PNG} 
   \caption{Temperature and Pressure Measurement System}
   \label{fig:example}
\end{figure}

\bigskip

\begin{figure}[h!] %  figure placement: here, top, bottom, or page
   \centering
   \includegraphics[width=5.5in]{lab_4a_grid.PNG} 
   \caption{Temperature and Pressure Measurement System Gauges and Charts}
   \label{fig:example}
\end{figure}

\newpage


In the second configuration shown in Figure 3, two harmonic signals with different frequencies and amplitudes were added together and the output was displayed on a chart in the time and frequency domains (see Figure 4), where the frequency domain information was attained by use of the FFT technique. The output was also passed through a first order Butterworth filter and plotted in the time and frequency domains. The order of the Butterworth filter was then changed to 3, and is shown in Figure 5. 
\bigskip
\bigskip
\bigskip
\bigskip

\begin{figure}[h!] %  figure placement: here, top, bottom, or page
   \centering
   \includegraphics[width=\linewidth]{lab_4b.PNG} 
   \caption{Signal Summing System utilizing FFT and Butterworth Filter}
   \label{fig:example}
\end{figure}

\newpage

\begin{figure}[h!] %  figure placement: here, top, bottom, or page
   \centering
   \includegraphics[width=6in]{lab_4b_1st_order.PNG} 
   \caption{Signal Summing System Time and Frequency Plots (1st Order)}
   \label{fig:example}
\end{figure}

\begin{figure}[h!] %  figure placement: here, top, bottom, or page
   \centering
   \includegraphics[width=6in]{lab_4b_3rd_order.PNG} 
   \caption{Signal Summing System Time and Frequency Plots (3rd Order)}
   \label{fig:example}
\end{figure}



\newpage

% Multisim figures
%\begin{figure}[h!] %  figure placement: here, top, bottom, or page
%   \centering
%   \includegraphics[width=5.5in]{non_inverting_multisim.PNG} 
%   \caption{Non-Inverting Amplifier (Multisim)}
%   \label{fig:example}
%\end{figure}
%
%\bigskip
%
%\begin{figure}[h!] %  figure placement: here, top, bottom, or page
%   \centering
%   \includegraphics[width=5in]{inverting_multisim.PNG} 
%   \caption{Inverting Amplifier (Multisim)}
%   \label{fig:example}
%\end{figure}
%
%\newpage
%
%\begin{figure}[h!] %  figure placement: here, top, bottom, or page
%   \centering
%   \includegraphics[width=4.5in]{follower_multisim.PNG} 
%   \caption{Follower Device (Signal Buffer) (Multisim)}
%   \label{fig:example}
%\end{figure}

\bigskip


\section*{\fontsize{12}{12}\selectfont \large Conclusion}
\addcontentsline{toc}{section}{Conclusion} % Add for each section
The exercises conducted in this lab served a learning exercise for Lab View. It is useful for mechanical engineers to know and understand how common systems such as measurement instruments actually work. It is shown that control systems can be modeled and analyzed in software. The systems created in this lab are common systems, and the creation of these systems in Lab View helps the student to further understand what is really happening in these systems. 


%\section*{\fontsize{12}{12}\selectfont \large References}

\begin{thebibliography}{2}

% Example
%\bibitem{Wagner}
%Ng, K., Wagner, S.W., Camelio, J., Emblom, W.J. (2010). ?Experimental Analysis of Micro Tube
%Hydroforming Process.? Transactions of NAMRC of SME, 38, 577-584.

\end{thebibliography}



%\section*{\fontsize{12}{12}\selectfont APPENDIX}

%\begin{table}[h!]
%  \caption{}
%  \includegraphics[width=\linewidth]{table1.png}
%\end{table}




\end{document}







----------------------------Templates-------------------------------

-------------------------Figure-----------------------

\begin{figure}[h!]  
  \centering
    \includegraphics[width=\linewidth]{**file**}
    \caption{Docking Station}
\end{figure}

---------------------------Table-----------------------
\begin{table}[ht]
\caption{Nonlinear Model Results} % title of Table
\centering % used for centering table
\begin{tabular}{c c c c} % centered columns (4 columns)
\hline\hline %inserts double horizontal lines
Case & Method\#1 & Method\#2 & Method\#3 \\ [0.5ex] % inserts table
%heading
\hline % inserts single horizontal line
1 & 50 & 837 & 970 \\ % inserting body of the table
2 & 47 & 877 & 230 \\
3 & 31 & 25 & 415 \\
4 & 35 & 144 & 2356 \\
5 & 45 & 300 & 556 \\ [1ex] % [1ex] adds vertical space
\hline %inserts single line
\end{tabular}
\label{table:nonlin} % is used to refer this table in the text
\end{table}



probably best to insert as an image from excel

\bigskip\\
\begin{table}[h!]
  \caption{}
  \includegraphics[width=\linewidth]{**file**}
\end{table}
\bigskip\\





-----------------------------Equations------------------------
-----------------------------Regular
\begin{equation}
a = b + c
\end{equation}

--------------------------------- Multiline
\begin{multline}
a = b + c + d + e + f
+ g + h + i + j \\
+ k + l + m + n + o
\end{multline}

-------------------------------Citations-------------------------
\bibitem{Author last name}
  Last, First., year of publication,
  article name, book(etc) name, from \\
  link goes here

----------------------------------other-----------------------------

equations:
http://moser-isi.ethz.ch/docs/typeset_equations.pdf

citations:
http://library.missouri.edu/engineering/about/guides/asme
https://www.asme.org/shop/proceedings/conference-publications/references